\rSec0[fiber]{API}

\indexlibrary{\idxhdr{experimental/fiber}}
\rSec1[fiber.synop]{Header \tcode{<experimental/fiber>} synopsis}

\begin{codeblock}
namespace std {
namespace experimental {
namespace fiber {
inline namespace v1 {

class fiber_context;

class unwind_exception;

void unwind_fiber(fiber_context&& other);

} // namespace v1
} // namespace fiber
} // namespace experimental
} // namespace std
\end{codeblock}

\rSec1[fiber.ctx]{Class \tcode{fiber_context}}

\indexlibrary{\idxcode{fiber_context}}
\pnum
Class \tcode{fiber_context} represents a lightweight thread of execution.

\begin{codeblock}
namespace std {
namespace experimental {
namespace fiber {
inline namespace v1 {

class fiber_context {
public:
    fiber_context() noexcept;

    template<typename Fn>
    explicit fiber_context(Fn&& fn);

    ~fiber_context();

    fiber_context(fiber_context&& other) noexcept;
    fiber_context& operator=(fiber_context&& other) noexcept;
    fiber_context(const fiber_context& other) noexcept = delete;
    fiber_context& operator=(const fiber_context& other) noexcept = delete;

    fiber_context resume() &&;
    template<typename Fn>
    fiber_context resume_with(Fn&& fn) &&;
    fiber_context resume_from_any_thread() &&;
    template<typename Fn>
    fiber_context resume_from_any_thread_with(Fn&& fn) &&;

    bool can_resume() noexcept;
    bool can_resume_from_any_thread() noexcept;

    explicit operator bool() const noexcept;
    void swap(fiber_context& other) noexcept;
};

} // namespace v1
} // namespace fiber
} // namespace experimental
} // namespace std
\end{codeblock}


\rSec2[fiber.ctx.cons]{\tcode{fiber_context} constructors}

\indexlibrary{\idxcode{fiber_context}!constructor}
\begin{itemdecl}
fiber_context() noexcept
\end{itemdecl}

\begin{itemdescr}
\pnum
\effects Constructs an object of class \tcode{fiber_context} which is an invalid \fiber.\\
\postconditions \tcode{!*this}
\end{itemdescr}

\indexlibrary{\idxcode{fiber_context}!constructor}
\begin{itemdecl}
template<typename Fn>
explicit fiber_context(Fn&& fn)
\end{itemdecl}

\begin{itemdescr}
\pnum
The constructor takes an invocable (function, lambda, object with \op) as
argument. The invocable must have signature as described in
\nameref{solution_gpub}.\\
This constructor template shall not participate in overload resolution unless
\tcode{Fn} is \emph{LvalueCallable} (23.14.13.2) for the argument type
\tcode{std::fiber_context\&\&} and the return type \fiber.\\
\begin{note}The entry-function \tcode{fn} is \emph{not} immediately entered. The
stack and any other necessary resources are created on construction, but
\tcode{fn} is not entered until \resume, \resumewith, \xtresume or
\xtresumewith is called.\end{note}\\
\begin{note}The entry-function \tcode{fn} passed to \fiber will be passed a
synthesized \fiber instance representing the suspended caller of \resume,
\resumewith, \xtresume or \xtresumewith.\end{note}
\end{itemdescr}

\indexlibrary{\idxcode{fiber_context}!constructor}
\begin{itemdecl}
fiber_context(fiber_context&& other) noexcept
\end{itemdecl}

\begin{itemdescr}
\pnum
moves underlying state to new \fiber\\
\postconditions \tcode{!*this} if \tcode{!other} before move; otherwise
\tcode{*this} and \tcode{!other}
\end{itemdescr}

\indexlibrary{\idxcode{fiber_context}!constructor}
\begin{itemdecl}
fiber_context(const fiber_context& other)=delete
\end{itemdecl}

\begin{itemdescr}
\pnum
copy constructor deleted
\end{itemdescr}



\rSec2[fiber.ctx.dtor]{\tcode{fiber_context} destructor}

\indexlibrary{\idxcode{fiber_context}!destructor}
\begin{itemdecl}
~fiber_context()
\end{itemdecl}

\begin{itemdescr}
\pnum
Destroys a \fiber instance. If this instance represents a fiber of execution
(\tcode{*this} returns \tcode{true}), then the fiber of execution is destroyed
too. Specifically, the stack is unwound by throwing \unwindex.\\
\begin{note}In a program in which exceptions are thrown, it is prudent to code a
fiber's \entryfn\xspace with a last-ditch \tcode{catch (...)} clause:
in general, exceptions must \emph{not} leak out of the \entryfn. However, since
stack unwinding is implemented by throwing an exception, a correct \entryfn
\tcode{try} statement must also \tcode{catch (std::unwind_exception const\&)}
and rethrow it.\end{note}
\end{itemdescr}



\rSec2[fiber.ctx.assign]{\tcode{fiber_context} assignment}

\indexlibrary{\idxcode{fiber_context}!equalop}
\begin{itemdecl}
fiber_context& operator=(fiber_context&& other) noexcept
\end{itemdecl}

\begin{itemdescr}
\pnum
assigns the state of \tcode{other} to \tcode{*this} using move semantics\\
\returns \tcode{*this}\\
\postconditions \tcode{!*this} if \tcode{!other} before move; otherwise
\tcode{*this} and \tcode{!other}
\end{itemdescr}

\indexlibrary{\idxcode{fiber_context}!equalop}
\begin{itemdecl}
fiber_context& operator=(const fiber_context& other)=delete
\end{itemdecl}

\begin{itemdescr}
\pnum
copy assignment operator deleted
\end{itemdescr}



\rSec2[fiber.ctx.modifiers]{\tcode{fiber_context} modifiers}

\indexlibrarymember{swap}{context}
\begin{itemdecl}
    void swap(fiber_context& other) noexcept
\end{itemdecl}

\begin{itemdescr}
\pnum
\effects Interchanges the targets of \tcode{*this} and \tcode{other}.
\end{itemdescr}



\rSec2[fiber.ctx.switch]{\tcode{fiber_context} switch}

\indexlibrarymember{resume}{context}%
\begin{itemdecl}
fiber_context resume() &&
\end{itemdecl}

\begin{itemdescr}
\pnum
\preconditions \tcode{*this} an if \canxtresume would return \tcode{false},
\currthread is the same as \lastthread\\
\effects suspends the active fiber, resumes fiber \tcode{*this}\\
\returns the returned instance represents the fiber that has been suspended in
order to resume the current fiber\\
\postcondition \tcode{!*this} and \currthread is the same as \lastthread\\
\throws \unwindex when, while suspended, the \fiber instance representing the
suspended fiber is destroyed
\end{itemdescr}

\indexlibrarymember{resume_with}{context}%
\begin{itemdecl}
template<typename Fn>
fiber_context resume_with(Fn&& fn) &&
\end{itemdecl}

\begin{itemdescr}
\pnum
\preconditions \tcode{*this)} an if \canxtresume would return \tcode{false},
\currthread is the same as \lastthread\\
\effects Suspends the active fiber, resumes fiber \tcode{*this} but calls
\tcode{fn()} in the resumed fiber (as if called by the suspended function).
\tcode{fn} is a invocable injected into resumed fiber.\\
These member function templates shall not participate in overload resolution
unless \tcode{Fn} is \emph{LvalueCallable} (23.14.13.2) for the argument type
\tcode{std::fiber_context\&\&} and the return type \fiber.\\
\returns the returned instance represents the fiber that has been suspended in
order to resume the current fiber\\
\postcondition \tcode{!*this}\\
\throws \unwindex when, while suspended, the \fiber instance representing the
suspended fiber is destroyed
\end{itemdescr}

\indexlibrarymember{resume_from_any_thread}{context}%
\begin{itemdecl}
fiber_context resume_from_any_thread() &&
\end{itemdecl}

\begin{itemdescr}
\pnum
\preconditions \tcode{*this} an if \canxtresume would return \tcode{false},
\currthread is the same as \lastthread\\
\effects suspends the active fiber, resumes fiber \tcode{*this}\\
\returns the returned instance represents the fiber that has been suspended in
order to resume the current fiber\\
\postcondition \tcode{!*this} and \currthread is the same as \lastthread\\
\throws \unwindex when, while suspended, the \fiber instance representing the
suspended fiber is destroyed
\end{itemdescr}

\indexlibrarymember{resume_from_any_thread_with}{context}%
\begin{itemdecl}
template<typename Fn>
fiber_context resume_from_any_thread_with(Fn&& fn) &&
\end{itemdecl}

\begin{itemdescr}
\pnum
\preconditions \tcode{*this} an if \canxtresume would return \tcode{false},
\currthread is the same as \lastthread\\
\effects Suspends the active fiber, resumes fiber \tcode{*this} but calls
\tcode{fn()} in the resumed fiber (as if called by the suspended function).
\tcode{fn} is a invocable injected into resumed fiber.\\
These member function templates shall not participate in overload resolution
unless \tcode{Fn} is \emph{LvalueCallable} (23.14.13.2) for the argument type
\tcode{std::fiber_context\&\&} and the return type \fiber.\\
\returns the returned instance represents the fiber that has been suspended in
order to resume the current fiber\\
\postcondition \tcode{!*this}\\
\throws \unwindex when, while suspended, the \fiber instance representing the
suspended fiber is destroyed
\end{itemdescr}

\resume, \resumewith, \xtresume or \xtresumewith can throw \emph{any} exception
if, while suspended:
\begin{itemize}
    \item some other fiber calls \resumewith or \xtresumewith to resume this
        suspended fiber
    \item the function \tcode{fn} passed to \resumewith or \xtresumewith -- or
        some function called by \tcode{fn} -- throws an exception
\end{itemize}
Any exception thrown by the function \tcode{fn} passed to \resumewith or
\xtresumewith, or any function called by \tcode{fn}, is thrown in the fiber
referenced by \tcode{*this} rather than in the fiber of the caller of
\resumewith or \xtresumewith.

The intent of the distinction between \resume and \xtresume, as between
\resumewith and \xtresumewith, is both for validation and for code auditing. If
an application only ever calls \resume and \resumewith, no fiber will ever be
resumed on a thread other than the one on which it was initially resumed.\\

The intent of the names \xtresume and \xtresumewith is to clarify the direction
in which cross-thread resumption occurs. \Currthread always directly resumes a
suspended fiber: control is passed into the suspended fiber, and the
currently-running fiber suspends. These method names mean that the fiber
represented by \tcode{*this} will be resumed whether or not it was last resumed
on \currthread.\\

\resume, \resumewith, \xtresume and \xtresumewith preserve the execution context
of the calling fiber. Those data are restored if the calling fiber is resumed.\\
A suspended \tcode{fiber\_context} can be destroyed. Its resources will be
cleaned up at that time.\\
The returned \tcode{fiber\_context} indicates via \tcode{*this} whether the
previous active fiber has terminated (returned from \entryfn).\\
Because \resume, \resumewith, \xtresume and \xtresumewith invalidate the
instance on which they are called, \emph{no valid \fiber instance ever
represents the currently-running fiber.} In order to express the invalidation
explicitly, these methods are rvalue-reference qualified.\\
When calling any of these methods, it is conventional to replace the
newly-invalidated instance -- the instance on which the method was was called
-- with the new instance returned by that call. This helps to avoid subsequent
inadvertent attempts to resume the old, invalidated instance.
\newline
An injected function \tcode{fn()} must have signature
\tcode{std::fiber_context fn(std::fiber_context\&\&)}. It will be passed a
synthesized \fiber instance representing the suspended caller of \resumewith or
\xtresumewith. The \fiber instance returned by \tcode{fn()} is, in turn, used as
the return value for the suspended function: \resume, \resumewith, \xtresume or
\xtresumewith.



\rSec2[fiber.ctx.ops]{\tcode{fiber_context} operations}

\indexlibrarymember{can_resume_from_any_thread}{context}%
\begin{itemdecl}
bool can_resume_from_any_thread() noexcept
\end{itemdecl}

\begin{itemdescr}
\pnum
query whether \currthread can resume the suspended \fiber instance by calling
\xtresume or \xtresumewith. The implementation must return \tcode{false} if the
suspended \fiber instance represents a fiber with a system-provided stack, and
\currthread is not that thread.\\
\returns \tcode{false} if \tcode{!*this} or if the stack used by the fiber was
provided by the operating system, and \currthread is not that thread; otherwise
\tcode{true}.\\
\begin{note}When \tcode{main()}, or the entry-function of a \thread, or any
function directly called by these, is suspended, a \fiber instance represents
that suspended fiber. You may resume that suspended fiber
\emph{on the same thread} using any of \resume, \resumewith, \xtresume or
\xtresumewith. Attempting to resume that suspended fiber from any other thread
is Undefined Behavior. \canxtresume returns \tcode{true} if \currthread is the
same as \lastthread, or if the \fiber instance represents a fiber explicitly
created by \fiber's constructor. \canxtresume is not marked \tcode{const}
because in at least one implementation, it requires an internal context switch.
\end{note}
\end{itemdescr}

\indexlibrarymember{can_resume}{context}%
\begin{itemdecl}
bool can_resume() noexcept
\end{itemdecl}

\begin{itemdescr}
\pnum
\returns \tcode{true} if \currthread is the same as \lastthread, or if
\tcode{*this} has not yet been resumed. When \canresume returns \tcode{true},
the \fiber instance may be resumed by \resume, \resumewith, \xtresume or
\xtresumewith.\\
\begin{note} \canresume is not marked \tcode{const} because in at least one
implementation, it requires an internal context switch.\end{note}
\end{itemdescr}



\rSec2[fiber.ctx.validity]{\tcode{fiber_context} validity}

\indexlibrarymember{operator bool}{context}%
\begin{itemdecl}
explicit operator bool() const noexcept
\end{itemdecl}

\begin{itemdescr}
\pnum
\returns \tcode{true} if \tcode{*this} represents a fiber of execution,
\tcode{false} otherwise.\\
A \fiber instance might not represent a valid fiber for any of a number of
reasons.
\begin{itemize}
    \item It might have been default-constructed.
    \item It might have been moved from.
    \item It might already have been resumed -- calling \resume, \resumewith,
          \xtresume or\\
          \xtresumewith invalidates the instance.
    \item The \entryfn\xspace might have voluntarily terminated the fiber by
          returning.
\end{itemize}
The essential points:
\begin{itemize}
    \item Regardless of the number of \fiber declarations, exactly one\\
          \fiber instance represents each suspended fiber.
    \item No \fiber instance represents the currently-running fiber.
\end{itemize}
\end{itemdescr}



\rSec1[fiber.unwind]{\tcode{fiber_context} unwinding}

\rSec2[fiber.unwind.func]{Function \tcode{unwind_fiber()}}

\indexlibrary{\idxcode{unwind_fiber}}
terminate the current running fiber, switching to the fiber represented by the
passed \fiber. This is like returning that \fiber from the \entryfn, but may be
called from any function on that fiber.

\begin{itemdecl}
[[ noreturn ]] void unwind_fiber(fiber_context&& other)
\end{itemdecl}

\begin{itemdescr}
\pnum
\preconditions \tcode{*other}\\
\returns does not return\\
\throws \unwindex\\
Throws \unwindex, binding the passed \fiber. The running fiber's first stack
entry catches \unwindex, extracts the bound \fiber and terminates the current
fiber by returning that \fiber. \tcode{other} is the \fiber to which to switch
once the current fiber has terminated
\end{itemdescr}



\rSec2[fiber.unwind.ex]{Class \tcode{unwind_exception}}

\indexlibrary{\idxcode{unwind_exception}}
\tcode{unwind_exception} is the exception used to unwind the stack referenced by
a \fiber being destroyed.
It is thrown by \unwindfib. \unwindex binds a \fiber referencing the fiber to
which control should be passed once the current fiber is unwound and destroyed.

\begin{codeblock}
namespace std {
namespace experimental {
namespace fiber {
inline namespace v1 {

class unwind_exception {
};

} // namespace v1
} // namespace fiber
} // namespace experimental
} // namespace std
\end{codeblock}
